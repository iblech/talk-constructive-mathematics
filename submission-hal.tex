\documentclass[a4paper,ngerman,10pt]{scrartcl}

\usepackage[utf8]{inputenc}
\usepackage[ngerman]{babel}
\usepackage[protrusion=true,expansion=true]{microtype}
\usepackage{amsthm,mathtools,booktabs,xcolor}

\usepackage{hyperref}
\hypersetup{colorlinks=true}

\usepackage{geometry}
\geometry{tmargin=2.3cm,bmargin=2.3cm,lmargin=3.9cm,rmargin=3.9cm}

\theoremstyle{plain}
\newtheorem*{prop}{Proposition}

\newcommand{\defeq}{\vcentcolon=}
\newcommand{\defequiv}{\vcentcolon\equiv}

\title{Konstruktive Mathematik, die Doppelnegationsübersetzung und Continuations}
\author{Ingo Blechschmidt}

\begin{document}

\begin{center}\large\textsf{\textbf{Konstruktive Mathematik, \\ die
Doppelnegationsübersetzung und Continuations}}

\normalsize Ingo Blechschmidt \texttt{<iblech@speicherleck.de>}\end{center}

\emph{Wie können wir das Minimum einer unendlichen Liste von natürlichen
Zahlen bestimmen?} Wenn wir die Liste nicht überschauen können, sind wir auf folgenden Trick
angewiesen. Wir bluffen und proklamieren einfach das erste Element der Liste als
Minimum. Durch Zufall könnte das sogar wirklich das kleinste Element der Liste
sein. Wenn uns aber jemand herausfordert und uns ein kleineres Element der
Liste präsentiert, so springen wir in der Zeit zurück und ändern die Geschichte
so, als ob wir von vornherein \emph{dieses} Element als Minimum proklamiert
hätten. Sollte auch dieser Bluff auffliegen, wiederholen wir das Prozedere.

Ist das geschummelt? Ja, in einem gewissen Sinn schon. Es steckt allerdings
mehr dahinter: Zum einen funktioniert dieses Verfahren "`von innerhalb der
Continuation-Monade betrachtet"' taddellos und kann in gewissen Situationen
durchaus nützlich sein. Obacht ist nur geboten, wenn man "`die
Continuation-Monade verlässt"'. Zum anderen entstammt das Verfahren nicht dem
übertriebenen Ehrgeiz eines Poker-Fans, sondern ergibt sich \emph{maschinell}
durch "`Extraktion algorithmischen Inhalts"' aus einem mathematischen Beweis
der einfachen Tatsache, dass jede unendliche Liste von natürlichen Zahlen ein
Minimum enthalten muss.

Im Vortrag werden wir alle Floskeln dieses Absatzes verstehen. Haskell-Code
wird die Sachverhalte illustrieren. Wir werden auch einen praktischen
Anwendungsfall aus der Kryptographie und Zahlentheorie diskutieren (je nach
Wunsch während des Vortrags oder durch ausführliche schriftliche Notizen), der
"`Bestimmung von modularen Inversen in Restklassenringen"', in der die
vorgestellten Techniken eine erhebliche Effizienzsteigerung ermöglichen.


\subsection*{Notwendiger Exkurs: Was ist konstruktive Mathematik?}

Den Unterschied zur gewöhnlichen, klassischen Mathematik begreift man am besten
an einem Beispiel. Eine Zahl heißt genau dann \emph{rational}, wenn sie sich als Bruch zweier
ganzer Zahlen schreiben lässt. Zum Beispiel ist~$\frac{21}{13}$ rational, die
Zahl~$\sqrt{2}$ ist es dagegen nicht. Nun kann man sich fragen, ob es
\emph{irrationale} Zahlen~$x$ und~$y$ gibt, die miteinander potenziert eine
\emph{rationale} Zahl als Ergebnis geben. Das folgende Argument zeigt, dass die
Antwort darauf positiv ist:
\begin{quote}
Zunächst ist die Zahl~$\sqrt{2}^{\sqrt{2}}$ entweder rational oder irrational.

\begin{itemize}
\item Im ersten Fall sind~$x \defeq \sqrt{2}$ und $y \defeq \sqrt{2}$ Zahlen
mit der gewünschten Eigenschaft.
\item Im zweiten Fall können wir~$x \defeq \sqrt{2}^{\sqrt{2}}$ und~$y \defeq
\sqrt{2}$ betrachten. Dann sind nämlich~$x$ und~$y$ irrational und die Potenz
$x^y = \sqrt{2}^{\sqrt{2} \cdot \sqrt{2}} = \sqrt{2}^2 = 2$ ist rational.
\end{itemize}
\end{quote}
Mit diesem Beweis kennen wir also die Antwort auf die Frage. Doch halt!
Tatsächlich sind wir immer noch nicht in der Lage, einem interessierten
Gegenüber ein Zahlenpaar mit den gewünschten Eigenschaften nennen zu können.
\emph{Der Beweis war unkonstruktiv.}

In konstruktiver Mathematik stellt man strengere Anforderungen an einen Beweis
-- so starke, dass man aus jedem konstruktiven Beweis einer Existenzbehauptung
einen Algorithmus ablesen kann, der das postulierte Objekt explizit berechnet.

Es stellt sich heraus, dass man, um dieses Ziel zu erreichen, auf genau ein
Axiom klassischer Logik verzichten muss: dem \emph{Axiom vom ausgeschlossenen
Dritten}. Dieses sagt in etwa aus, dass jede Aussage stimmt oder nicht stimmt;
im obigen Beweis ging es gleich im ersten Schritt ein.

Im Vortrag werden wir verstehen, welchen Standpunkt man einnehmen muss, damit
der Verzicht auf dieses Axiom nicht völlig verrückt erscheint (ist es nicht
einfach offensichtlich wahr?). Außerdem werden wir die
\emph{Doppelnegationsübersetzung} kennenlernen, die folgende fundamentale
Eigenschaft hat: Genau dann gibt es einen klassischen Beweis einer Aussage
-- einen, in dem das Axiom vom ausgeschlossenen Dritten verwendet werden darf
-- wenn es einen konstruktiven Beweis der übersetzten Aussage gibt.


\subsection*{Die Curry--Howard-Korrespondenz}

Die Curry--Howard-Korrespondenz identifiziert \emph{logische Aussagen} mit
\emph{Typen} und \emph{Beweise von Aussagen} mit \emph{Termen geeigneten Typs}.
Unpräzise als Motto formuliert besagt sie, dass Programmieren und konstruktive
Mathematik Betreiben ein und dasselbe sind. Unter dieser Korrespondenz
entspricht beispielsweise der triviale Beweis der Behauptung "`aus $A$ folgt
$A$"' der Identitätsfunktion~\texttt{id :: A $\to$ A}. Der triviale Beweis der
Behauptung "`wenn $A$ und $B$, dann insbesondere $A$"' entspricht der
Funktion~\texttt{fst :: (A,B) $\to$ A}. Das werden wir im Vortrag noch genauer
thematisieren.

\begin{center}\begin{tabular}{ll}
  \toprule
  Logik & Programmierung \\\midrule
  Aussage $A$ & Typ \texttt{A} \\
  Konstruktiver Beweis $p$ von $A$ & Term \texttt{p :: A} \\
  Konjunktion $A \wedge B$ ("`und"') & Produkttyp \texttt{(A,B)} \\
  Disjunktion $A \vee B$ ("`oder"') & Summentyp \texttt{Either A B} \\
  Implikation $A \Rightarrow B$ ("`wenn, dann"') & Funktionstyp \texttt{A $\to$ B} \\
  \bottomrule
\end{tabular}\end{center}

Mit der Curry--Howard-Korrespondenz steckt also in jedem konstruktiven Beweis
ein Programm; in diesem Sinn hat jeder konstruktiver Beweis "`algorithmischen
Inhalt"'. Diese Korrespondenz ist aber auf konstruktive Beweise beschränkt. Nur
dank der Doppelnegationsübersetzung kann man auch noch in klassischen Beweisen
algorithmischen Inhalt entdecken: indem man die Korrespondenz auf ihre
Doppelnegationsübersetzung anwendet.

Welches Pendant auf der Seite der Programmierung gehört eigentlich zur
Dop\-pel\-ne\-ga\-tions\-über\-set\-zung? \emph{Faszinierenderweise ist das gerade die beim
Compilerbau oft verwendete Continuation-Passing-Style-Transformation.} Die
CPS-Transformation, mit der man etwa beliebige Kontrollstrukturen recht leicht
implementieren kann, wurde in den 60er-Jahren entwickelt. Die
Doppelnegationsübersetzung wird seit den 30er-Jahren untersucht. Die
Curry--Howard-Korrespondenz vereint also zwei scheinbar unterschiedliche
Konzepte aus unterschiedlichen Epochen und unterschiedlichen Teilgebieten der
Mathematik und Informatik.


\subsection*{Fallbeispiel: Minimum einer unendlichen Liste}

Ist~\texttt{[x0, x1, x2, ...]} eine unendliche Liste von natürlichen Zahlen, so
ist klar: In dieser Liste ist eines der Elemente das kleinste, das Minimum. Wenn man
die Situation genauer untersucht, erkennt man jedoch, dass der Beweis dieser
Behauptung das Axiom vom ausgeschlossenen Dritten verwendet und daher nicht
konstruktiv ist. In Übereinstimmung mit der Curry--Howard-Korrespondenz können
wir auch keine Funktion~\texttt{minimum :: [Nat] $\to$ Nat} schreiben, die das
Minimum einer übergebenen unendlichen Liste bestimmen würde.

Was passiert, wenn wir auf den klassischen Beweis die
Doppelnegationsübersetzung anwenden, so einen konstruktiven Beweis erhalten und
dann diesen mit der Curry--Howard-Korrespondenz in ein Programm umwandeln? Dann
erhalten wir eine Funktion vom Typ \[\texttt{minimum :: [Nat] $\to$ Cont r (Nat,
Nat $\to$ Cont r ())}.\] Dabei ist~\texttt{Cont r} die Con\-ti\-nuation-Monade. Der
Rückgabewert~\texttt{(n, f)} von~\texttt{minimum xs} enthält an erster Stelle
das Minimum der Liste und an zweiter Stelle eine Funktion, die die Minimalität
von~\texttt{n} bezeugt: Übergibt man ihr einen Folgenindex~\texttt{i}, so
antwortet sie mit einem Beweis der Behauptung~\texttt{n $\leq$ xs!!i}. (Um das
treu in Haskell abbilden zu können, bräuchte man \emph{abhängige Typen}. Da es
die in Haskell nicht gibt, muss der triviale Typ~\texttt{()} zusammen mit einem
sozialen Versprechen als Ersatz herhalten.)

Was macht diese Funktion? \emph{Sie setzt genau das eingangs beschriebene
Verfahren um.}

\renewcommand{\thefootnotemark}{}
\footnotetext{Originalliteratur:
\href{http://ecommons.library.cornell.edu/bitstream/1813/6991/1/90-1151.pdf}{Chetan Murthy,
Extracting Constructive Content from Classical Proofs, 1990}. Popularisierungen: in
Artikeln von
\href{http://citeseerx.ist.psu.edu/viewdoc/summary?doi=10.1.1.49.3492}{Thierry Coquand},
\href{http://okmij.org/ftp/Computation/lem.html}{Oleg Kiselyov},
\href{https://lukepalmer.wordpress.com/2007/10/26/computing-without-a-middle/}{Luke Palmer},
\href{http://blog.sigfpe.com/2008/06/drugs-kate-moss-and-intuitionistic.html}{Dan Piponi (sigfpe)},
\href{http://homepages.inf.ed.ac.uk/wadler/papers/dual/dual.pdf}{Philip Wadler} und
\href{http://blog.ezyang.com/2013/04/a-classical-logic-fairy-tale/}{Edward Z. Yang}.
Empfehlenswert ist in diesem Zusammenhang auch ein Buchkapitel von
\href{http://math.andrej.com/wp-content/uploads/2014/03/real-world-realizability.pdf}{Andrej Bauer}.}

\end{document}


In der Informatik untersucht man seit den 60er-Jahren das Konzept der
\emph{Continuations}. Im Compilerbau ist dabei die
\emph{Continuation-Passing-Style-Transformation} eine wichtige Hilfe, da man
nach Anwendung dieser Transformation alle möglichen Kontrollstrukturen leicht
implementieren kann.
In der Logik gibt es seit den 30er-Jahren die
\emph{Doppelnegationsübersetzung}. Die ändert eine Aussage so ab, dass die
übersetzte Aussage genau dann "`in konstruktiver Mathematik"' gilt, wenn die
ursprüngliche Aussage "`in klassischer Mathematik"' gilt.

Die gefeierte \emph{Curry--Howard-Korrespondenz} besagt in etwa, dass
Programmieren und konstruktive Mathematik Betreiben ein und dasselbe sind; aus
jedem konstruktiven Beweis kann man einen Algorithmus extrahieren. Unter diesem
Gesichtspunkt sind CPS-Transformation und Doppelnegationsübersetzung
faszinierenderweise nur zwei unterschiedliche
Sichtweisen desselben Konzepts. Hinzu kommt, dass man damit auch aus Beweisen
\emph{klassischer} Mathematik algorithmischen Inhalt extrahieren kann. Dieser ist
längst nicht so direkt wie der Inhalt konstruktiver Beweise, sondern von einer
monadischen Natur. Wie er genau aussieht und wozu er gut ist, werden wir im
Vortrag verstehen. Haskell-Code wird dabei die Sachverhalte illustrieren.


Was macht diese Funktion? \emph{Sie blufft.} Sie speichert zunächst die
aktuelle Continuation. Dann proklamiert sie willkürlich das erste Element der
Liste als Minimum und gibt dieses als erste Komponente~\texttt{n} zurück. Als
Zeugen der Minimalität gibt sie eine Funktion~\texttt{f} zurück, die bei Aufruf
von~\texttt{f i} zunächst prüft, ob zufälligerweise~\texttt{n $\leq$ xs!!i} gelten
sollte. Wenn ja, kann der Bluff noch gehalten werden. Aber wenn nein, dann ist
mit~\texttt{xs!!i} ein kleineres Element der Liste gefunden worden. In diesem
Fall nimmt sie die zuvor gespeicherte Continuation, macht den vorherigen Bluff
ungeschehen und ändert die Geschichte so, dass der Rückgabewert~\texttt{n} nun
nicht mehr das erste Element der Liste ist, sondern~\texttt{xs!!i}. Wenn auch
der neue Bluff auffliegen sollte, geht es auf dieselbe Art und Weise weiter.

Von innerhalb der Continuation-Monade betrachtet macht diese
Funktion~\texttt{minimum} genau das, was sie soll. Vorsicht ist nur geboten,
wenn man die Continuation-Monade verlässt -- dann ist die Fehlerkorrektur durch
Zurückspringen nicht mehr möglich.


Ein praktischer Anwendungsfall liegt in der Bestimmung von Inversen in
so genannten Restklassenringen, welche in der Kryptographie und Zahlentheorie wichtig
sind. Aus der in klassischer Logik trivialen Beobachtung "`jedes Polynom ist
entweder reduzibel oder irreduzibel"' folgt mit den vorgestellten Techniken
eine erhebliche Effizienzsteigerung. Dieses Beispiel kann bei Interesse noch
während des Vortrags diskutiert werden und wird ansonsten ausführlich
schriftlich erklärt.


\subsection*{In Kürze}

Es ist spannend, dass CPS-Transformation und Doppelnegationsübersetzung ein und
dasselbe sind. Auch in klassischen Beweisen steckt algorithmischer Inhalt,
allerdings ist dieser von einer nicht ganz direkten, monadischen Natur. Im
Vortrag werden wir das genauer verstehen und praktische Anwendungen
kennenlernen.
