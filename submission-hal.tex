\documentclass[a4paper,ngerman]{scrartcl}

\usepackage[utf8]{inputenc}
\usepackage[ngerman]{babel}
\usepackage[protrusion=true,expansion=true]{microtype}
\usepackage{amsthm,mathtools,booktabs}

\theoremstyle{plain}
\newtheorem*{prop}{Proposition}

\newcommand{\defeq}{\vcentcolon=}
\newcommand{\defequiv}{\vcentcolon\equiv}

\title{Konstruktive Mathematik, \\ die Doppelnegationsübersetzung \\ und Continuations}
\author{Ingo Blechschmidt}

\begin{document}

\maketitle

\section*{Worum geht es?}

In der Informatik untersucht man seit den XXXer Jahren das Konzept der
\emph{Continuations}. Im Compilerbau ist dabei die
\emph{Continuation-Passing-Style-Transformation} eine wichtige Hilfe, da man
nach Anwendung dieser Transformation alle möglichen Kontrollstrukturen leicht
implementieren kann.

In der Logik gibt es die \emph{Doppelnegationsübersetzung}. Die änderte eine
Aussage so ab, dass die übersetzte Aussage genau dann "`in konstruktiver
Mathematik"' gilt, wenn die ursprüngliche Aussage "`in klassischer Mathematik"'
gilt.

Die gefeierte \emph{Curry--Howard-Korrespondenz} besagt in etwa, dass
Programmieren und konstruktive Mathematik Betreiben ein und dasselbe sind; aus
jedem konstruktiven Beweis kann man einen Algorithmus extrahieren. Unter diesem
Gesichtspunkt sind die CPS-Transformation und Doppelnegationsübersetzung
faszinierenderweise nur zwei unterschiedliche
Sichtweisen desselben Konzepts, und damit kann man auch aus Beweisen
klassischer Mathematik algorithmischen Inhalt extrahieren.


\section*{Was ist konstruktive Mathematik?}

Eine Zahl heißt genau dann \emph{rational}, wenn sie sich als Bruch zweier
ganzer Zahlen schreiben lässt. Zum Beispiel ist~$\frac{21}{13}$ rational, die
Zahl~$\sqrt{2}$ ist es dagegen nicht. Nun kann man sich fragen, ob es
\emph{irrationale} Zahlen~$x$ und~$y$ gibt, die miteinander potenziert eine
\emph{rationale} Zahl als Ergebnis geben. Das folgende Argument zeigt, dass die
Antwort darauf positiv ist:
\enlargethispage{2em}

\begin{quote}
Die Zahl~$\sqrt{2}^{\sqrt{2}}$ ist entweder rational oder irrational.

\begin{itemize}
\item Im ersten Fall sind~$x \defeq \sqrt{2}$ und $y \defeq \sqrt{2}$ Zahlen
mit der gewünschten Eigenschaft.
\item Im zweiten Fall können wir~$x \defeq \sqrt{2}^{\sqrt{2}}$ und~$y \defeq
\sqrt{2}$ betrachten. Dann ist nämlich~$x^y = \sqrt{2}^{\sqrt{2} \cdot
\sqrt{2}} = \sqrt{2}^2 = 2$ rational.
\end{itemize}
\end{quote}

Mit diesem Beweis kennen wir also die Antwort auf die Frage. Doch halt!
Tatsächlich sind wir immer noch nicht in der Lage, einem interessierten
Gegenüber ein Zahlenpaar mit den gewünschten Eigenschaften nennen zu können.
\emph{Der Beweis war unkonstruktiv.}

In konstruktiver Mathematik stellt man strengere Anforderungen an einen Beweis
-- so starke, dass man aus jedem konstruktiven Beweis einer Existenzbehauptung
einen Algorithmus ablesen kann, der das postulierte Objekt explizit berechnet.

Es stellt sich heraus, dass man, um dieses Ziel zu erreichen, auf genau ein
Axiom klassischer Logik verzichten muss: dem \emph{Axiom vom ausgeschlossenen
Dritten}. Dieses sagt in etwa aus, dass jede Aussage stimmt oder nicht stimmt;
im obigen Beweis ging es gleich im ersten Schritt ein.

Im Vortrag werden wir verstehen, welchen Standpunkt man einnehmen muss, damit
der Verzicht auf dieses Axiom nicht völlig verrückt erscheint (ist es nicht
einfach offensichtlich wahr?). Außerdem werden wir die
\emph{Doppelnegationsübersetzung} kennenlernen, die folgende fundamentale
Eigenschaft hat: Genau dann gibt es einen klassischen Beweis einer Aussage
-- einen, in dem das Axiom vom ausgeschlossenen Dritten verwendet werden darf
-- wenn es einen konstruktiven Beweis der übersetzten Aussage gibt.


\section*{Die Curry--Howard-Korrespondenz}

Die Curry--Howard-Korrespondenz identifiziert \emph{logische Aussagen} mit
\emph{Typen} und \emph{Beweise von Aussagen} mit \emph{Termen geeigneten Typs}.
Unter dieser Korrespondenz entspricht beispielsweise der triviale Beweis der
Behauptung "`aus $A$ folgt $A$"' der Identitätsfunktion~\texttt{($\lambda$x ->
x) :: A $\to$ A}. Der triviale Beweis der Behauptung "`wenn $A$ und $B$, dann
insbesondere $A$"' entspricht der Funktion~\texttt{fst :: (A,B) $\to$ A}.

\begin{center}\begin{tabular}{ll}
  \toprule
  Logik & Programmierung \\\midrule
  Aussage $A$ & Typ \texttt{A} \\
  Konstruktiver Beweis $p$ von $A$ & Term \texttt{p :: A} \\
  Konjunktion: $A$ und $B$ & Produkttyp \texttt{(A,B)} \\
  Disjunktion: $A$ oder $B$ & Summentyp \texttt{Either A B} \\
  Implikation: Wenn $A$, dann $B$ & Funktionstyp \texttt{A $\to$ B} \\
  \bottomrule
\end{tabular}\end{center}

Mit der Curry--Howard-Korrespondenz steckt also in jedem konstruktiven Beweis
ein Programm; in diesem Sinn hat jeder konstruktiver Beweis "`algorithmischen
Inhalt"'.

\end{document}

Nun gibt es eine faszinierende Verbindung zwischen der
Doppelnegationsübersetzung und der in der Informatik wohlbekannten
Continuation-Passing-Style-Transformation: In einem gewissen Sinn sind sie ein
und dasselbe. Das ist ein wunderschöner Aspekt des "computational
trinitarianism".

Der Vortrag wird in diese Thematik einführen und anhand von Beispielcode in
Haskell illustrieren. Scheibar unmögliche Aufgaben wie das Minimum einer
unendlichen Liste von natürlichen Zahlen zu finden werden damit -- in einem
gewissen monadischen Sinn -- möglich. Der Vortrag setzt keine Vorkenntnisse in
formaler Logik oder konstruktiver Mathematik voraus, wohl aber gewisse
allgemeine Vertrautheit mit mathematischen Fragestellungen und mathematischer
Sprache.
