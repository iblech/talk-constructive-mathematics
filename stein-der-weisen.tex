\documentclass[16pt]{scrartcl}

\usepackage[utf8]{inputenc}
\usepackage[ngerman]{babel}

\usepackage[T1]{fontenc}
\usepackage{libertine}

\usepackage[expansion=true,protrusion=true]{microtype}

\setlength\parskip{\bigskipamount}
\setlength\parindent{0pt}
\pagestyle{empty}

\usepackage[left=2.5cm,top=2cm,right=2.5cm,bottom=1cm]{geometry}

\begin{document}

\begin{center}
{\large \textbf{Ein Märchen über klassische Logik}}

\emph{Laut und deutlich sprechen. Danke! :-)}
\end{center}

\textbf{Erzähler.}
Vor langer, langer Zeit begab sich im fernen, fernen Curry\-land folgende
Geschichte. Eines Tages holte die Königin des Landes und aller Haskellistas und
Lambdroiden ihren Haus- und Hof-Phi\-lo\-so\-phen zu sich.

\textbf{Königin.}
Philosoph! Ich habe folgenden Auftrag an dich: Beschaffe mir den Stein der
Weisen, oder alternativ finde heraus, wie man mithilfe des Steins unbegrenzt
Gold herstellen kann!

\textbf{Philosoph.}
Aber meine Königin! Ich habe nichts Brauchbares studiert! Wie soll ich diese
Aufgabe erfüllen?

\textbf{Königin.}
Das ist mir egal! Wir sehen uns morgen wieder. Erfüllst du deine Aufgabe nicht,
sollst du gehängt werden. Oder wir hacken deinen Kopf ab und verwenden ihn als
Cricket-Ball.

\textbf{Erzähler.}
Nach einer schlaflosen Nacht voller Sorgen wurde der Philosoph erneut zur
Königin berufen.

\textbf{Königin.}
Nun! Was hast du mir zu berichten?

\textbf{Philosoph.}
Ich habe es tatsächlich geschafft, herauszufinden, wie man den Stein verwenden
könnte, um unbegrenzt Gold herzustellen. Aber nur ich kann dieses Verfahren
durchführen, Eure Hoheit.

\textbf{Königin.}
Nun gut, dann sei es so!

\textbf{Erzähler.}
Und so vergingen die Jahre, in denen sich der Philosoph in Sicherheit wähnte
und die Angst vor Cricket-Schlägern langsam verlor. Die Königin suchte nun
selbst nach dem Stein, aber solange sie ihn nicht fand, hatte der Philosoph
nichts zu befürchten.

\textbf{Erzähler.}
Doch eines Tages passierte das Unfassbare: Die Königin hatte den Stein
gefunden! Und lies prompt den Philosophen zu sich rufen.

\textbf{Königin.}
Philosoph, sieh! Ich habe den Stein der Weisen gefunden, hier! Nun erfülle du
deinen Teil der Abmachung! \emph{[übergibt den Stein]}

\textbf{Philosoph.}
Danke. Ihr hattet von mir verlangt, Euch den Stein der Weisen zu beschaffen
oder herauszufinden, wie man mit ihm unbegrenzt Gold herstellen kann. Hier habt
Ihr den Stein der Weisen. \emph{[übergibt den Stein zurück]}

\end{document}
