\documentclass[12pt,utf8,notheorems,compress]{beamer}
\usepackage{etex}

\usepackage[english]{babel}

\usepackage{tikz}
\usetikzlibrary{decorations.pathmorphing,arrows}
\usepackage{mathtools}
\usepackage{booktabs}
\usepackage{array}
\usepackage{ragged2e}
\usepackage{multicol}
\usepackage{tabto}
\usepackage{xstring}

\usepackage[protrusion=true,expansion=false]{microtype}

\setlength\parskip{\medskipamount}
\setlength\parindent{0pt}

\renewcommand{\U}{\mathcal{U}}
\newcommand{\NN}{\mathbb{N}}
\newcommand{\id}{\mathrm{id}}
\newcommand{\ZZ}{\mathbb{Z}}
\newcommand{\RR}{\mathbb{R}}
\newcommand{\Id}{\mathrm{Id}}
\newcommand{\fst}{\mathsf{fst}}
\newcommand{\snd}{\mathsf{snd}}
\newcommand{\inv}{\mathsf{inv}}
\newcommand{\refl}{\mathsf{refl}}
\newcommand{\ap}{\mathsf{ap}}
\renewcommand{\succ}{\mathsf{succ}}
\newcommand{\seg}{\mathsf{seg}}
\newcommand{\base}{\mathsf{base}}
\newcommand{\lloop}{\mathsf{loop}}
\newcommand{\ua}{\mathsf{ua}}
\newcommand{\code}{\mathsf{code}}
\newcommand{\surf}{\mathsf{surf}}
\newcommand{\merid}{\mathsf{merid}}
\newcommand{\N}{\mathsf{N}}
\renewcommand{\S}{\mathsf{S}}
\newcommand{\Cyl}{\mathrm{Cyl}}
\newcommand{\IsContr}{\mathsf{IsContr}}
\newcommand{\IsMereProp}{\mathsf{IsMereProp}}
\newcommand{\IsSet}{\mathsf{IsSet}}
\newcommand{\IsEquiv}{\mathsf{IsEquiv}}
\newcommand{\LEM}{\mathsf{LEM}}
\newcommand{\UIP}{\mathsf{UIP}}
\newcommand{\fib}{\mathsf{fib}}
\newcommand{\List}{\mathsf{List}}
\newcommand{\defeq}{\vcentcolon=}
\newcommand{\defeqv}{\vcentcolon\equiv}
\newcommand{\ct}{%
  \mathchoice{\mathbin{\raisebox{0.5ex}{$\displaystyle\centerdot$}}}%
             {\mathbin{\raisebox{0.5ex}{$\centerdot$}}}%
             {\mathbin{\raisebox{0.25ex}{$\scriptstyle\,\centerdot\,$}}}%
             {\mathbin{\raisebox{0.1ex}{$\scriptscriptstyle\,\centerdot\,$}}}
}

\title{The double negation translation}
\author[XXX]{\vspace{-1em}\\\includegraphics[scale=0.3]{torus.png} \\[0.5em] Ingo Blechschmidt \\[-0.3em] {\scriptsize June 3rd, 2015}}
\date{June 3d, 2015}

\usetheme{Warsaw}
\usecolortheme{seahorse}
%\usefonttheme{default}?
%\usepackage{kurier}?
\usefonttheme{serif}
%\usepackage{libertine}?
\usepackage{mathpazo}
\useinnertheme{rectangles}

\setbeamertemplate{title page}[default][colsep=-1bp,rounded=false,shadow=false,bg=white]
\setbeamertemplate{frametitle}[default][colsep=-2bp,rounded=false,shadow=false,center]

\setbeamertemplate{headline}{}
\setbeamertemplate{navigation symbols}{}

\newcommand{\floatbox}[3]{%
  \raisebox{0pt}[0pt][0pt]{%
    \begin{picture}(0,0)(#1,#2)#3\end{picture}\leavevmode%
  }%
}

\newcommand{\backupstart}{
  \newcounter{framenumberpreappendix}
  \setcounter{framenumberpreappendix}{\value{framenumber}}
}
\newcommand{\backupend}{
  \addtocounter{framenumberpreappendix}{-\value{framenumber}}
  \addtocounter{framenumber}{\value{framenumberpreappendix}} 
}

\newcommand*\oldmacro{}%
\let\oldmacro\insertshorttitle%
\renewcommand*\insertshorttitle{%
  \oldmacro\hfill\insertframenumber\,/\,\inserttotalframenumber\hfill}

\newcommand{\hil}[1]{{\usebeamercolor[fg]{item}{\textbf{#1}}}}

\newcommand{\img}[3]{\begin{center}\includegraphics[scale=#1]{#2}\\\scriptsize#3\end{center}}
%\newcommand{\imageslide}[3]{\frame{\frametitle{#1}\img{#2}{#3}}}

\IfSubStr{\jobname}{\detokenize{nonotes}}{
  \setbeameroption{hide notes}
}{
  \setbeameroption{show notes}
}
\setbeamertemplate{note page}[plain]

\newenvironment{changemargin}[2]{%
  \begin{list}{}{%
    \setlength{\topsep}{0pt}%
    \setlength{\leftmargin}{#1}%
    \setlength{\rightmargin}{#2}%
    \setlength{\listparindent}{\parindent}%
    \setlength{\itemindent}{\parindent}%
    \setlength{\parsep}{\parskip}%
  }%
  \item[]}{\end{list}}

\begin{document}

\frame{\titlepage}

\frame[t]{\frametitle{Outline}\scriptsize\begin{itemize}\item[]\tableofcontents\end{itemize}}

\note{\fontsize{8pt}{9.6}\selectfont
  \begin{center}\large\textbf{Abstract}\end{center}

  \begin{changemargin}{2.5em}{2.5em}
    \justifying
    Constructive mathematicians don't use the law of excluded middle, which
    approximately says that for any proposition~$P$, either~$P$ is true or~$\neg P$
    is true. Several advantages emerge from this rejection, for instance one
    can mechanically extract algorithms from constructive proofs of existence
    statements and rigorously work with non-standard \emph{dream axioms} which are
    plainly false in classical mathematics, such as \emph{any function is
    smooth}.

    For communicating with classicial mathematicians, constructive
    mathematicians can employ the \emph{double-negation translation}. This
    device associates to any formula a translated formula in such a way that
    a given formula holds classically if and only if its translation holds
    constructively.

    The talk will give an introduction to these topics and discuss the
    intriguing relationship of the double-negation translation with the
    well-known con\-tin\-u\-a\-tion-pas\-sing transformation: In some sense, they are
    the same. This is a beautiful facet of \emph{computational trinitarianism}.

    For the first part of the talk, no background in formal logic or
    constructive mathematics is required. For the second part of the talk,
    one should be vaguely familiar with the continuation-passing
    transformation.
  \end{changemargin}
}

\end{document}

1. Constructive mathematics
   * LEM
   * (Informal) BKH interpretation
   * Applications

2. The double negation translation
   * Proof of neg neg (phi v neg phi)
   * Game-theoretical intepretation
   * The translation and the fundamental result

3. Relation to continuations
   * ...

4. Outlook
   * CPS transformation = Yoneda embedding
   * Box-translation for arbitrary modal operators Box
   * negneg-sheafification?
