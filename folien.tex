\documentclass[12pt]{beamer}

\usepackage{ucs}
\usepackage[utf8x]{inputenc}

\usepackage[ngerman]{babel}

\usepackage{amsmath,amssymb}
%\usepackage[framed,amsmath,thmmarks,hyperref]{ntheorem}

%\usepackage[small,nohug]{diagrams}
%\diagramstyle[labelstyle=\scriptstyle]

%\usepackage[protrusion=true,expansion=false]{microtype}

%\usepackage{lmodern}
\usepackage{tabto}

%\usepackage[natbib=true,style=numeric]{biblatex}
%\usepackage[babel]{csquotes}
%\bibliography{lit}

%\usepackage{hyperref}

\setlength\parskip{\medskipamount}
\setlength\parindent{0pt}

%\theoremseparator{:}
\theoremstyle{plain}  %nonumberplain
%\newtheorem{beh}{Behauptung}
\newtheorem{proposition}{Proposition}
%\newtheorem{kor}{Korollar}
%\newtheorem{satz}{Satz}
%\newtheorem{lemma}{Lemma}
%\newtheorem{hilfsaussage}{Hilfsaussage}
%\theorembodyfont{\normalfont}
\newtheorem{axiom}{Axiom}
%\newtheorem{defnprop}{Definition/Proposition}
%\newtheorem{bem}{Bemerkung}
%\newtheorem{bsp}{Beispiel}
%\theoremsymbol{\ensuremath{\openbox}}
%\newtheorem{proof}{Beweis}
%\newtheorem{defn}{Definition}

\newcommand{\lra}{\longrightarrow}
\newcommand{\lhra}{\ensuremath{\lhook\joinrel\relbar\joinrel\rightarrow}}
\newcommand{\thlra}{\relbar\joinrel\twoheadrightarrow}

\newcommand{\Z}{\mathbb{Z}}
\newcommand{\C}{\mathbb{C}}
\newcommand{\N}{\mathbb{N}}
\newcommand{\Hom}{\mathrm{Hom}}
\newcommand{\Spur}[1]{\operatorname{Spur}#1}
\newcommand{\SpurDyn}[1]{\operatorname{Spur}\left(#1\right)}
\newcommand{\rank}[1]{\operatorname{rank}#1}
\newcommand{\Ker}[1]{\operatorname{ker}#1}
\newcommand{\Bild}[1]{\operatorname{im}#1}
\newcommand{\sgn}[1]{\operatorname{sgn}#1}
\newcommand{\id}{\mathrm{id}}
\newcommand{\Aut}[1]{\operatorname{Aut}(#1)}
\newcommand{\GL}[1]{\operatorname{GL}(#1)}
\newcommand{\freist}{\_{}\_{}}

\renewcommand{\P}{\mathcal{P}}
\newcommand{\1}{\mathbf{1}}
\renewcommand{\_}{\mathpunct{.}\,}

\newcommand{\XXX}[1]{\textcolor{red}{#1}}

%\newarrow{Equals}=====

\title{Intuitionistische Logik}
\author{Ingo Blechschmidt}
%\subtitle{Vektorbündel, K-Theorie und \\ charakteristische Klassen}
\date{Sommerakademie in Neubeuern \\ \ \\ ??. August 2012}

%\usetheme{Warsaw}  %Warsaw, Berkeley?
\usetheme{Warsaw}
\useoutertheme{split}
\usecolortheme{seahorse}
\usefonttheme{serif}
\usepackage{palatino}
\useinnertheme{rectangles}
%\usepackage{bookman}
%\setbeamercovered{transparent}

\setbeamertemplate{navigation symbols}{}
\setbeamertemplate{footline}{}
\setbeamertemplate{headline}{}

\beamertemplateboldcenterframetitle
\setbeamerfont{frametitle}{size={\Large}}

\begin{document}

\frame{\titlepage}
\frame[t]{\tableofcontents}

\section{Einführung}

\subsection{Das Axiom vom ausgeschlossenen Dritten}
\frame[t]{\frametitle{Das Axiom vom ausgeschlossenen Dritten}
  \begin{axiom}[klassisch]
    Eine jede Aussage stimmt oder stimmt nicht.
  \end{axiom}

  \begin{itemize}
    \item intuitionistische Logik $:=$
          klassische Logik ohne LEM
  \end{itemize}
}

\section{Eine andere mathematische Welt}

% XXX: Fehlt Intro.
% insbes. erklären, dass wir durchaus an not(phi) ^ not(not(phi)) glauben.
\subsection{Leere und nichtleere Mengen}
\frame[t]{\frametitle{Leere und nichtleere Mengen}
  Sei~$X$ eine Menge.
  
  \emph{Frage:} Sind die beiden folgenden Aussagen
  äquivalent?
  \begin{enumerate}
    \item $X$ ist \emph{nicht leer}, d.\,h. $X \neq \emptyset$.
    \item $X$ ist \emph{bewohnt}, d.\,h. $\exists x \in X$.
  \end{enumerate}

  \pause
  Aussage~{\insertenumlabel} ist stärker: Man kann explizit ein Element~$x$ der
  Menge angeben.
}

\subsection{Teilmengen der Eins}
\frame[t]{\frametitle{Teilmengen der Eins}
  \begin{definition}Die \emph{Menge der Wahrheitswerte~$\Omega$} ist
  die Menge aller Teilmengen der Menge
  \[ \1 := \{ \star \}, \]
  also
  \[ \Omega := \P(\1) = \{ U \subseteq \1 \}. \]
  \end{definition}

  \begin{itemize}
    \item \emph{Frage:} Wie viele Elemente enthält~$\Omega$?
    \pause
    \item Es gibt auch
    \begin{align*}
      \Omega_{\text{entscheidbar}} &:= \{ p \in \Omega \,|\, \text{$\star \in p$
      oder $\star \not\in p$} \}, \\
      \Omega_{\text{klassisch}} &:= \{ p \in \Omega \,|\, \neg \neg(\star \in
      p) \Rightarrow \star \in p \}.
    \end{align*}
    In welcher Beziehung stehen diese zu~$\Omega$?
  \end{itemize}
}

% XXX fehlt: topologische Interpretation

\subsection{Minima von Mengen natürlicher Zahlen}
\frame[t]{\frametitle{Minima von Mengen natürlicher Zahlen}
  Klassisch gilt: Jede bewohnte Menge natürlicher Zahlen besitzt ein Minimum.

  \emph{Frage:} Stimmt das auch intuitionistisch?
}

\subsection{Endliche Mengen}
\frame[t]{\frametitle{Endliche Mengen}
  \begin{definition}
    Sei~$X$ eine Menge. $X$ heißt\ldots
    \begin{itemize}
      \item \ldots\emph{endlich} \tabto{3.4cm} $:\Leftrightarrow$
        $\exists n \in \N\_ \exists f{:}\, [n] \to X\_ \text{$f$ bijektiv}$.
      \item \ldots\emph{endlich indiziert} \tabto{3.4cm} $:\Leftrightarrow$
        $\exists n \in \N\_ \exists f{:}\, [n] \to X\_ \text{$f$ surjektiv}$.
    \end{itemize}
    Dabei ist~$[n] = \{ m \in \N \,|\, m < n \} = \{ 0, 1, \ldots, n-1 \}$.
  \end{definition}

  \emph{Frage:} Sind Teilmengen endlicher (endl. indizierter) Mengen wieder
  endlich (endl. indiziert)?
}


\section{Eleganzassistenz}

\subsection{Unendlichkeit der Primzahlen}
\frame[t]{\frametitle{Unendlichkeit der Primzahlen}
  \begin{theorem}[schwache Form]Die Menge~$\mathbb{P}$ der Primzahlen,
  \[ \mathbb{P} := \{ 2, 3, 5, 7, 11, 13, \ldots \} \subseteq \N, \]
  ist nicht endlich.\end{theorem}
  \pause

  \begin{theorem}[starke Form]Sei~$S$ eine endliche Menge von Primzahlen. Dann gibt es
  eine weitere Primzahl, die nicht in~$S$ enthalten ist.\end{theorem}
}

\subsection{Übungsaufgabe in Linearer Algebra I}
\frame[t]{\frametitle{Übungsaufgabe in Linearer Algebra I}
  \begin{proposition}
    Sei~$f{:}\, X \to Y$ eine Abbildung und sei
    \[ \renewcommand{\arraystretch}{1.3}\begin{array}{@{}rrcl@{}}
      \varphi{:} &\P(Y)&\longrightarrow& \P(X), \\
      & U &\longmapsto& f^{-1}[U] := \{ x \in X \,|\, f(x) \in U \}
    \end{array} \]
    die zugehörige Urbildoperation. Dann ist~$f$ genau dann surjektiv,
    wenn~$\varphi$ injektiv ist.
  \end{proposition}

  \emph{Zum Knobeln:} Auch für die Rückrichtung gibt es einen direkten Beweis.
}

\end{document}
